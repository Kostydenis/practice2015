%!TEX root = main.tex
\subsubsection{Загрузка базы DEF-кодов}

Для загрузки базы с удаленного адреса была создана программа на языке программирования Python (Листинг~\ref{lst:dwnldData}).

В процессе анализа работы сервиса был выявлен формат запроса для получения базы с url \url{http://mtt.ru/defcodes}. Он имеет следующий вид (Листинг~\ref{lst:request})

\lstcode{codes/request.py}{0.6}{\small}{Формат запроса}{request}

Отправление запроса происходит методом \texttt{post}, который содержится в модуле \texttt{requests}. Данный оператор принимает два аргумента: адрес и параметры запроса, указанные выше.

Ответ от сервера приходит в формате JSON, который содержит два поля:
\begin{enumerate*}
	\item \texttt{'status'} --- состояние ответа (success --- данные успешно получены; error --- возникла ошибка; captcha --- слишком много запросов с IP-адреса и неверно указан проверочный код);
	\item \texttt{'resultHTML'} --- база DEF-кодов в формате HTML-таблицы.
\end{enumerate*}
 
После получения успешного ответа от сервера, программа сохраняет локальную копию в файл \texttt{base.html} и начинает анализ таблицы. Для данной операции используется модуль BeautifulSoup, которая позволяет работать с сущностями таблицы как с объектами. Проанализированные данные программа сохраняет в JSON-файл \texttt{base.json}, с которым в последствии работает основное приложение.

\lstcode{codes/dwnldData.py}{0.6}{\small}{Python-скрипт для загрузки базы DEF-кодов}{dwnldData}

Готовый скрипт упакован с помощью утилит "py2app" для Mac OS X и "py2exe" для Windows. Пример сценария для сборки Mac OS X версии представлен ниже (Листинг~\ref{lst:setupmac})

\lstcode{codes/setup_mac.py}{0.6}{\small}{Python-скрипт для сборки Mac OS X версии}{setupmac}
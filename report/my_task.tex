%!TEX root = main.tex
\section{Индивидуальное задание}

Основной задачей производственной практики являлась разработка программы для получения списка операторов с соответствующими им DEF-кодами и диапазонами номеров и выводе его в удобном для дальнейшего использования экспертом формате.

\subsection{Постановка задачи}

Задачу можно разделить на несколько подзадач:

\begin{enumerate*}
	\item сформировать запрос с необходимой информацией и получить полный список с url \url{http://mtt.ru/defcodes};
	\item выбрать из полученного списка необходимые область, оператора и def-код;
	\item преобразовать диапазон в заданный шаблон;
	\item Результат работы программы вывести на экран, в текстовые файлы или файл электронных таблиц.
\end{enumerate*}

Результат работы программы должен быть выведен в формате, представленном ниже.

\begin{itemize*}
	\item Текстовые файлы:
		В каталоге с датой и временем вывода\footnote{Уникальное имя каталога необходимо для предотвращения конфликтов при выводе} создать каталоги с названием области, в каждом из которых сформировать .txt-файл с названием оператора, с вложенными шаблонами диапазонов для конкретных области и оператора.

	\item Файл таблицы:
		В каталоге с датой и временем вывода создать файл электронной таблицы с названиями листов, соответствующим областям, с вложенными шаблонами диапазонов для всех операторов конкретной области.

\end{itemize*}

%!TEX root = main.tex
\subsection{Используемые инструменты}

При разработке данной программы использовалось несколько языков программирования, фреймворков и библиотек.

\subsubsection{Javascript, Node.js, node-webkit, JSON}

Основная часть программы, объединяющая все модули приложения и обеспечивающая взаимодействие с пользователем разработана с использованием языка программирования Javascript.

Выбор данной технологии был обусловлен возможностью разработки кросс-платформенных приложений. Данное свойство приложения обеспечивается фреймворком node-webkit\cite{nwjs}. Он обеспечивает среду выполнения для HTML+Javascript приложений. Преимуществом данной технологии является отстутсвие требований к установленному программному обеспечению пользователя. Все необходимые компоненты присутствуют в дистрибутиве программы.

К недостаткам данного подхода можно отнести большой размер распространяемой программы (115 Мб для Mac OS X и 95 Мб для Windows).

Для взаимодействия с операционной системой (использование файлов, выполнение сторонних приложений, и~т.~п.) используется фреймворк Node.js\cite{nodejs}.

Для хранения различных данных на диске, а также для обмена информацией между частями программы был выбран формат JSON\cite{json}. Выбор был обусловлен тем, что несмотря на тот факт, что стандарт основан на языке Javascript, он получил широкое распространение и, как следствие, поддержку большим числом языков программирования.

\subsubsection{HTML5, CSS3, SCSS}
Для верстки графического интерфейса пользователя использовался язык разметки HTML5.

Стили были разработаны с использованием препроцессора SCSS\cite{scss}, которые в свою очередь были скомпилированы  в CSS3 и подключены к HTML-файлу. В дополнение к стилям программы в SCSS-файл были интегрированы стили сторонних библиотек.

\subsubsection{Python 3.4.3}

В связи с политикой безопасности web-стандартов\cite{mdncrossdomain}, в языке Javascript существует ограничение на доступ к данным, находящимся на другом url. Так как загрузка базы DEF-кодов осуществлялась с домена, доступа к которому осуществлялся исключительно через API провайдера, для решения данной задачи был создан Python-скрипт\cite{pythonoff}.

\subsubsection{Git}

В качестве системы управления версиями использовался интернет-сервис GitHub\cite{github}.

Данный сервис позволяет хранить все версии приложения, а также распространять открытый исходный код.



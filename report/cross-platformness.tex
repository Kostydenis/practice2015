%!TEX root = main.tex
\subsubsection{Реализация кросс-платформенности}

Весь код приложения является кросс-платформенным, имея различия лишь в дистрибутиве контейнера node-webkit.

Некоторые особенности операционной системы учтены с помощью условных конструкций проверки свойства \texttt{os.platform()} (\texttt{sys.platform} в Python): для Mac OS X функция возвращает строку \texttt{``darwin''}, для Windows --- \texttt{``win32''}. Например, в листинге~\ref{lst:updatebase()} представлена реализация в \texttt{js}-программе --- функция обновления базы. Для выполнения данной функции в различных средах необходимо запускать разные команды, \texttt{open} для Mac OS X и \texttt{start} для Windows.

\lstcode{codes/updateBase().js}{0.6}{\small}{updateBase().js}{updatebase()}

В листинге~\ref{lst:setpath} представлена реализация в \texttt{python}-скрипте. Эта часть кода выбирает путь до базы исходя из операционной системы, в которой запущен скрипт, что связано с особенностями упаковки сценария.

\lstcode{codes/setpath.py}{0.6}{\small}{Выбор пути до файла базы в \texttt{python}-скрипте}{setpath}

